Lectura recomendada: Secciones 1.1, 1.2, 1.3 de [R16].

Lista de ejercicios: 1.1.i, 1.1.ii, 1.1.iii (i), 1.2.ii, 1.2.iii, 1.2.iv, 1.2.v, 1.3.i, 1.3.ii, 1.3.iii, 1.3.viii, 1.3.ix, 1.3.x.

{\bf Soluciones: }\footnote{Soluciones proporcionadas por Alejandro García}

\begin{Ej}\label{ej:1.1.i}
   1.1.i


   {\bf (i)} Consideramos un morfismo $f \colon x \rightarrow y$. Demuestra que si existe un par de morfismos $g,h \colon \rightrightarrows x$ tal que $gf=\id{x}$ y $fh=\id{y}$,  consecuentemente $g=h$ y $f$ es un isomorfismo.

   $f \colon x \rightarrow y$, $g \colon y \rightarrow x$, $h \colon y \rightarrow x$; $gf=\id{x}$, $fh=\id{y}$ 

\begin{equation*}
gfh = g \left( fh\right) = g \id{y} = \left(gf\right) h = \id{x}h \Rightarrow g=h
\end{equation*}


  Por tanto $fh=\id{y}$ y $hf=\id{x}$ de esta forma $f$ es isomorfismo. 
  
  {\bf (ii)} Demuestra que un morfismo como máximo pude tener un único morfismo inverso.  

  Dado $f \colon x \rightarrow y$, un par de isomorfismos que invierten $f$ son dos morfismos $g_1,g_2 \colon y \rightarrow x $ tal que $g_1f=g_2=id{x}$ y existan $h_1,h_2 \colon x \rightarrow y$ tal que $g_1 h_1=g_2h_2=\id{x}$ y $h_1g_1=h_2g_2=\id{y}$. 

  Por tanto, aplicando {\bf (i)} obtenemos $h_1=f=h_2$ y como $g_1f=\id{x}$, $fg_2=\id{y}$ tenemos que $g_1=g_2$.



\end{Ej}


\begin{Ej}\label{ej:1.1.ii}
1.1.ii  

  Dada una categoría $\mathsf{C}$. Demuestra que la colección de isomorfismos en $\mathsf{C}$ definen una subcategoría, llamada grupoide máximal dentro de $\mathsf{C}$.


\end{Ej}



\begin{Ej}\label{ej:1.1.ii_i}
  1.1.ii 

  Por cualquier categoría $\mathsf{C}$ y cualquier objeto $c\in \mathsf{C}$, demuestra:

  {\bf (i)} Existe una categoría $c/\mathsf{C}$ cuyos objetos son morfismo $f \colon c \rightarrow x$ con dominio $c$ y un morfismo de $f\colon c \rightarrow x$ a $g\colon c \rightarrow y$ es un morfismo $h \colon x \rightarrow y$ de forma que 
\[
  \begin{tikzcd}
    & c \arrow[dl,"f"swap] \arrow[dr,"g"] \\
    x \arrow[rr,"h"swap] & & y
  \end{tikzcd}
\]
  es un diagrama conmutativo ($g=hf$).

\end{Ej}
